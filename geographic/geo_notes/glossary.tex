
\newglossaryentry{Orthorectification}{
    name=Orthorectification,
    description={ A method of correcting an image to align with real-world coordinates on a map. 
                  This involves measuring the exact location of the image center as well as 
                  the camera angle.  This is followed by the computation of the camera calibration 
                  parameters to remove camera and lens distortions.  Finally, you may terrain
                  induced distortions using DEM data.\index{Orthorectification}}
}

\newglossaryentry{Georectification}{
    name={Georectification},
    description={ A method of stretching and warping an image to align with another map projectin or spatial 
                  data in GIS.  This is comparable to Google Earth and other systems which implement overlays. 
                  If an image is rectified, Ground Control Points (GCP) can be used to create a transformation which 
                  aligns one image to the GIS data.  This is different from orthorectification as well because
                  it is assumed that the image is already orthorectified. Georectification just changes the 
                  projection and/or coordinate system.\index{Georectification}}
}

\newglossaryentry{Georeference}{
    name=Georeference,
    description={Same as Georectification \index{Georeference}}
}

\newglossaryentry{Aerotriangulation}{
    name=Aerotriangulation,
    description={  The process of assigning ground control values to points on a block of photographs
                   by determining the relationship between the photographs and known ground control 
                   points.\index{Aerotriangulation}}
}

\newglossaryentry{Bundle Adjustment}{
    name={Bundle Adjustment},
    description={ The process of simulaneously refining 3D coordinates derived from multiple viewpoints.  
                  This requires that the user has multiple 3d coordinates measured from multiple image
                  pairs. This is often solved with Levenberg-Marquardt.}
}

\newglossaryentry{Boresight}{
    name={Boresight},
    description={ Boresight is the physical mounting angles between an IMU and a digital camera. Basically, if the IMU
                  defines a flight axis,  the Boresight defines the angles from the axis of which the camera is pointing.}
}


\newglossaryentry{Focal Length}{
    name={Focal Length},
    description={  The distance between the focal point and the image plane. This is relevant as the focal length determines
                   attributes such as the clarity of the image and the depth of field.  It is an essential part of camera 
                   calibration.}
}

\newglossaryentry{Camera Origin}{
    name={Camera Origin},
    description={  The center of the camera in world coordinates.  In reality, this should be
                   the latitude, longitude, and elevation of the camera if described in geodetic coordinates. }
    symbol={$P_O$}
}

\newglossaryentry{Principle Point}{
     name={Principle Point},
     description={ The intersection of the ray which defines the center of the camera view with the image plane. This is the
                   physical location in world coordinates of the center of the image view.  This subtracted with the Camera Origin define
                   the vector which points ``straight ahead". },
     symbol={$P_P$}
}

\newglossaryentry{Principle Ground Point}{
    name={Principle Ground Point},
    description={Using the ray defined by the Camera Origin and the line $P_OP_P$, the intersection of this line with the
                 surface of the Earth creates a coordinate on the surface which is located in the principle point of the image. 
                 This is relevant because images can only give you 3D information up to scale and depth.  This will provide the depth 
                 attribute.}
    symbol={$P_G$}
}

