%%%%%%%%%%%%%%%%%%%%%%%%%%%%%%%%%%%%%%%%%%%%%%%%%%%%%%%%%%%%%%%%%%%%%%%%
%%%%%%%%%%%%%%%%%%%%%% Simple LaTeX CV Template %%%%%%%%%%%%%%%%%%%%%%%%
%%%%%%%%%%%%%%%%%%%%%%%%%%%%%%%%%%%%%%%%%%%%%%%%%%%%%%%%%%%%%%%%%%%%%%%%

%%%%%%%%%%%%%%%%%%%%%%%%%%%%%%%%%%%%%%%%%%%%%%%%%%%%%%%%%%%%%%%%%%%%%%%%
%% NOTE: If you find that it says                                     %%
%%                                                                    %%
%%                           1 of ??                                  %%
%%                                                                    %%
%% at the bottom of your first page, this means that the AUX file     %%
%% was not available when you ran LaTeX on this source. Simply RERUN  %%
%% LaTeX to get the ``??'' replaced with the number of the last page  %%
%% of the document. The AUX file will be generated on the first run   %%
%% of LaTeX and used on the second run to fill in all of the          %%
%% references.                                                        %%
%%%%%%%%%%%%%%%%%%%%%%%%%%%%%%%%%%%%%%%%%%%%%%%%%%%%%%%%%%%%%%%%%%%%%%%%

%%%%%%%%%%%%%%%%%%%%%%%%%%%% Document Setup %%%%%%%%%%%%%%%%%%%%%%%%%%%%

% Don't like 10pt? Try 11pt or 12pt
\documentclass[10pt]{article}

% This is a helpful package that puts math inside length specifications
\usepackage{calc}
\usepackage{url}

% Simpler bibsection for CV sections
% (thanks to natbib for inspiration)
\makeatletter
\newlength{\bibhang}
\setlength{\bibhang}{1em}
\newlength{\bibsep}
 {\@listi \global\bibsep\itemsep \global\advance\bibsep by\parsep}
\newenvironment{bibsection}
    {\minipage[t]{\linewidth}\list{}{%
        \setlength{\leftmargin}{\bibhang}%
        \setlength{\itemindent}{-\leftmargin}%
        \setlength{\itemsep}{\bibsep}%
        \setlength{\parsep}{\z@}%
        }}
    {\endlist\endminipage}
\makeatother

% Layout: Puts the section titles on left side of page
\reversemarginpar

%
%         PAPER SIZE, PAGE NUMBER, AND DOCUMENT LAYOUT NOTES:
%
% The next \usepackage line changes the layout for CV style section
% headings as marginal notes. It also sets up the paper size as either
% letter or A4. By default, letter was used. If A4 paper is desired,
% comment out the letterpaper lines and uncomment the a4paper lines.
%
% As you can see, the margin widths and section title widths can be
% easily adjusted.
%
% ALSO: Notice that the includefoot option can be commented OUT in order
% to put the PAGE NUMBER *IN* the bottom margin. This will make the
% effective text area larger.
%
% IF YOU WISH TO REMOVE THE ``of LASTPAGE'' next to each page number,
% see the note about the +LP and -LP lines below. Comment out the +LP
% and uncomment the -LP.
%
% IF YOU WISH TO REMOVE PAGE NUMBERS, be sure that the includefoot line
% is uncommented and ALSO uncomment the \pagestyle{empty} a few lines
% below.
%

%% Use these lines for letter-sized paper
\usepackage[paper=letterpaper,
            %includefoot, % Uncomment to put page number above margin
            marginparwidth=1.2in,     % Length of section titles
            marginparsep=.05in,       % Space between titles and text
            margin=0.75in,               % 1 inch margins
            includemp]{geometry}

%% Use these lines for A4-sized paper
%\usepackage[paper=a4paper,
%            %includefoot, % Uncomment to put page number above margin
%            marginparwidth=30.5mm,    % Length of section titles
%            marginparsep=1.5mm,       % Space between titles and text
%            margin=25mm,              % 25mm margins
%            includemp]{geometry}

%% More layout: Get rid of indenting throughout entire document
\setlength{\parindent}{0in}

%% This gives us fun enumeration environments. compactitem will be nice.
\usepackage{paralist}

%% Reference the last page in the page number
%
% NOTE: comment the +LP line and uncomment the -LP line to have page
%       numbers without the ``of ##'' last page reference)
%
% NOTE: uncomment the \pagestyle{empty} line to get rid of all page
%       numbers (make sure includefoot is commented out above)
%
\usepackage{fancyhdr,lastpage}
\pagestyle{fancy}
%\pagestyle{empty}      % Uncomment this to get rid of page numbers
\fancyhf{}\renewcommand{\headrulewidth}{0pt}
\fancyfootoffset{\marginparsep+\marginparwidth}
\newlength{\footpageshift}
\setlength{\footpageshift}
          {0.5\textwidth+0.5\marginparsep+0.5\marginparwidth-2in}
\lfoot{\hspace{\footpageshift}%
       \parbox{4in}{\, \hfill %
                    \arabic{page} of \protect\pageref*{LastPage} % +LP
%                    \arabic{page}                               % -LP
                    \hfill \,}}

% Finally, give us PDF bookmarks
\usepackage{color,hyperref}
\definecolor{darkblue}{rgb}{0.0,0.0,0.3}
\hypersetup{colorlinks,breaklinks,
            linkcolor=darkblue,urlcolor=darkblue,
            anchorcolor=darkblue,citecolor=darkblue}

%%%%%%%%%%%%%%%%%%%%%%%% End Document Setup %%%%%%%%%%%%%%%%%%%%%%%%%%%%


%%%%%%%%%%%%%%%%%%%%%%%%%%% Helper Commands %%%%%%%%%%%%%%%%%%%%%%%%%%%%

% The title (name) with a horizontal rule under it
%
% Usage: \makeheading{name}
%
% Place at top of document. It should be the first thing.
\newcommand{\makeheading}[1]%
        {\hspace*{-\marginparsep minus \marginparwidth}%
         \begin{minipage}[t]{\textwidth+\marginparwidth+\marginparsep}%
                {\large \bfseries #1}\\[-0.15\baselineskip]%
                 \rule{\columnwidth}{1pt}%
         \end{minipage}}

% The section headings
%
% Usage: \section{section name}
%
% Follow this section IMMEDIATELY with the first line of the section
% text. Do not put whitespace in between. That is, do this:
%
%       \section{My Information}
%       Here is my information.
%
% and NOT this:
%
%       \section{My Information}
%
%       Here is my information.
%
% Otherwise the top of the section header will not line up with the top
% of the section. Of course, using a single comment character (%) on
% empty lines allows for the function of the first example with the
% readability of the second example.
\renewcommand{\section}[2]%
        {\pagebreak[2]\vspace{1.3\baselineskip}%
         \phantomsection\addcontentsline{toc}{section}{#1}%
         \hspace{0in}%
         \marginpar{
         \raggedright \scshape #1}#2}

% An itemize-style list with lots of space between items
\newenvironment{outerlist}[1][\enskip\textbullet]%
        {\begin{itemize}[#1]}{\end{itemize}%
         \vspace{-.6\baselineskip}}

% An environment IDENTICAL to outerlist that has better pre-list spacing
% when used as the first thing in a \section
\newenvironment{lonelist}[1][\enskip\textbullet]%
        {\vspace{-\baselineskip}\begin{list}{#1}{%
        \setlength{\partopsep}{0pt}%
        \setlength{\topsep}{0pt}}}
        {\end{list}\vspace{-.6\baselineskip}}

% An itemize-style list with little space between items
\newenvironment{innerlist}[1][\enskip\textbullet]%
        {\begin{compactitem}[#1]}{\end{compactitem}}

% An environment IDENTICAL to innerlist that has better pre-list spacing
% when used as the first thing in a \section
\newenvironment{loneinnerlist}[1][\enskip\textbullet]%
        {\vspace{-\baselineskip}\begin{compactitem}[#1]}
        {\end{compactitem}\vspace{-.6\baselineskip}}

% To add some paragraph space between lines.
% This also tells LaTeX to preferably break a page on one of these gaps
% if there is a needed pagebreak nearby.
\newcommand{\blankline}{\quad\pagebreak[2]}

% Uses hyperref to link DOI
\newcommand\doilink[1]{\href{http://dx.doi.org/#1}{#1}}
\newcommand\doi[1]{doi:\doilink{#1}}


%%%%%%%%%%%%%%%%%%%%%%%% End Helper Commands %%%%%%%%%%%%%%%%%%%%%%%%%%%

%%%%%%%%%%%%%%%%%%%%%%%%% Begin CV Document %%%%%%%%%%%%%%%%%%%%%%%%%%%%

\begin{document}
\makeheading{Marvin L.~Smith}

\section{Contact Information}
%
% NOTE: Mind where the & separators and \\ breaks are in the following
%       table.
%
% ALSO: \rcollength is the width of the right column of the table
%       (adjust it to your liking; default is 1.85in).
%
\newlength{\rcollength}\setlength{\rcollength}{3.5in}%
%
\begin{tabular}[t]{@{}p{\textwidth-\rcollength}p{\rcollength}}
1750 Wewatta St, Unit 1300, 
& \textsl{E-mail:} 
\href{mailto:marvin\_smith1@me.com}{marvin\_smith1@me.com}\\
Denver, CO, 80202
& \textsl{Cell:} (775) 997-4261 \\          
& \textsl{LinkedIn:} \href{https://www.linkedin.com/in/marvin-smith-b8b99420/}{link}\\
& \textsl{Github:} \href{http://github.com/marvins}{http://github.com/marvins}\\
\end{tabular}

%**************************************************************
%                      WORK EXPERIENCE
%**************************************************************
\section{Work Experience}

\href{http://northropgrumman.com}{
    \textbf{Senior Principal Software Engineer, Launch and Missile Defense Systems}}\\
    Northrop Grumman Corporation, Boulder, CO\\
    January 2023 - Present
    \begin{outerlist}
        \item \textbf{Software Engineering Architect - FORGE Legacy SSP, 2023 - Present}\\
        \begin{innerlist}
            \item Algorithm SME on the Future Operationally Resilient Ground Evolution (FORGE) program.  
                  Responsible for building high throughput, low latency, and scientifically accurate imagery and telemetry processing algorithms
                  for the SBIRS Missile-Defense system. \href{https://scitec.com/scitec-inc-awarded-45-8-million-contract-for-continued-development-of-forge-legacy-sbirs-ssp/}{\underline{link}}\\
        \end{innerlist}
    \end{outerlist}
%SNC
\href{http://www.sncorp.com}{
     \textbf{Principal Software Engineer, ISR, Aviation, and Security (IAS) Division}}\\
     Sierra Nevada Corporation, Englewood, CO\\
     January 2012 - January 2023
     \begin{outerlist}
         \item \textbf{Systems Engineering, Integration, and Test (SEIT) Lead, 2021 - Present}\\
         \begin{innerlist}
            \item Long-Wave IR, Wide-Area Motion-Imagery (WAMI) Platform.
            \item Led a team of 10+ Systems Engineers, Scientists, and Program Coordinators.
            \item Responsible for the integration, testing, and fielding of the platform.
            \begin{innerlist}
                \item Developed the test plan and procedures to verify system and subsystem requirements. 
                \item Coordinated and led flight test activities to verify requirements.
                \item Responsible for managing and tracking delivery of customer CDRLs.
            \end{innerlist}
            \item I wrote a significant amount of code, as well as trained other non-SW engineers to write code.  This was done to help the Systems Engineering 
                  team to more efficiently perform analysis, increasingly automate testing, and to satisfy program objectives with appropriate rigor.\\
         \end{innerlist}
         
         \item \textbf{Lead Developer, 2016 - 2021}
         \item \textbf{Software Engineer, 2012 - 2016} 
         \begin{innerlist}
            \item Led the Image-Processing software team on the Gorgon-Stare Wide-Area Surveillance System.
            \item Designed and implemented ortho-image rendering algorithms for use on airborne systems as well as HPC systems.
            \item Developed camera calibration process using feature detection and Bundle-Adjustment algorithms.
            \item Developed significant quantities of mathematical software with a heavy emphasis on Numerical Methods, Statistics, Linear Algebra, and Machine-Learning.
            \item Designed and implemented image stabilization algorithms.
            \item Implemented or utilized many Image-Processing algorithms. Examples include Fourier transforms, Sharpening/Smoothing/Enhancement operations, Segmentation, and Inpainting.
            \item Created GIS data (still imagery, motion imagery, vector data) in formats
                  such as Pixia NUI, NITF, KML, MPEG/H264, Shapefiles, etc.
            \item Assisted in the development of new camera models for multi-camera sensor payloads.
            \item Designed, written, and traced system and software requirements.  I've helped create and modify requirements to fulfill customer needs against technical constraints.
            \item Significant test experience.  I've written test plans, test cards, and mapped test activities against system requirements. I can summarize a complex activity such as measuring Ground-Location-Accuracy (GLA) of an ortho-image into a process which test personnel can replicate and sign-off on.
         \item I configured and deployed a significant portion of the initial Continuous-Integration, unit-test, and deployment infrastructure 
               required by our team.
         \end{innerlist}
         \item Primary fields of expertise are in remote sensing and high-performance computing. 
         \item I have an active DOD security clearance.\\
     \end{outerlist}

%USAF
\href{http://www.nv.ngb.army.mil/air/index.cfm}
     {\textbf{Staff Sergeant, Nevada Air National Guard}},\\
     Guidance and Control Shop, Avionics Flight, 152 MXS, Nevada Air National Guard\\
     August 2006 - August 2012
     \begin{outerlist}
     \item Instrumentation and Flight Controls Journeyman (2A553B).
     \item Maintenance and repair of USAF C-130 Guidance and Control Systems. Systems
           include Engine Instrumentation, Autopilot, Fuel Quantity,
           Digital Flight Data Recorder, and Flight Director.
     \item Operation Enduring Freedom Deployment, Bagram AB, Afghanistan\\     
           August 2009 - December 2009
     \item Operation Iraqi Freedom Deployment, Talil AB, Iraq\\               
           October 2007 - January 2008\\
     \end{outerlist}
%NASA

\href{http://ti.arc.nasa.gov/tech/asr/intelligent-robotics/}
     {\textbf{Intern, Intelligent Robotics Group, NASA Ames Research Center}},\\
      Mountain View, CA \hspace{48mm}
      
      \begin{outerlist}
      \item June 2011 - August 2011
         \begin{innerlist}
         \item[] Developed a crater detection algorithm for use in planetary
               surface characterization and terrain reconstruction. Primary use
               is for the alignment of LIDAR data from the LRO Satellite
               to images taken from the Apollo 15 and 17 missions.
         \end{innerlist}
      
      \item June 2010 - August 2010
      \begin{innerlist}
      \item[] Developed algorithms for the open-source NASA Vision Workbench library
            which removes outliers from stereo reconstructions of the lunar surface.\\
      \end{innerlist}
      \end{outerlist}


%CVL
\href{http://www.cse.unr.edu/CVL/}
     {\textbf{Undergraduate Assistant, Computer Vision Lab}},\\
     Department of Computer Science and Engineering, University of Nevada, Reno\\
     August 2010 - May 2011, August 2011 - January 2012
     \begin{outerlist}
     \item Teaching Assistant
        \begin{innerlist}
        \item CS 302, Data Structures: CS 474, Image Processing: CS 485, Computer Vision\\
        \end{innerlist}
     \end{outerlist}


%**************************************************************
%                     TECHNICAL SKILLS
%*************************************************************
\section{Technical Skills}
\begin{outerlist}

\item{\textbf{Programming Languages}:} C++, Python, Matlab, \LaTeX{}, PowerShell, and Bash.

\item{\textbf{Software Experience}:} 
   \begin{outerlist}
   \item Image Processing : OpenCV, Matlab, NASA Vision Workbench, scikit
   \item GIS APIs : GDAL, ArcGIS, GeoServer, libLAS (lidar), GeographicLib (EGM96/geoid APIs), some ArcPy
   \item Distributed Computing : MPI, IBVerbs, libRDMA, and BSD Sockets
   \item Scientific APIs:  Eigen, Ceres, GMP, NumPy/SciPy, and Pandas
   \end{outerlist}

\item{\textbf{Open-Source Projects Contributed:}}
    \begin{outerlist}
    \item OpenCV
    \begin{innerlist}
        \item Wrote the update to the imgcodec API to allow using GDAL to load GIS raster format.
    \end{innerlist}
    \item NASA Vision Workbench
    \begin{innerlist}
        \item Wrote underlying implementations of the Hough transform, Integral-Image, and other common CV methods. 
    \end{innerlist}    
    \end{outerlist}

\item{\textbf{System Administration}:} Configuration Management deployment (git, svn, Artifactory), constructing software build systems (Jenkins), and some networking to setup software labs.

\end{outerlist}


%**************************************************************
%                     PUBLICATIONS
%*************************************************************
\section{Publications}
\begin{outerlist}

\item Technical Reviewer\\
Prateek Joshi, ``OpenCV with Python By Example", Packt Publishing, October 2015, 
\href{https://www.packtpub.com/application-development/opencv-python-example}{link}

\item Technical Reviewer\\
Garcia, Aranda, Suarez, Tercero, ``Learning Image Processing with OpenCV", Packt Publishing, March 2015, 
\href{https://www.packtpub.com/application-development/learning-image-processing-opencv}{link}

\item Marvin Smith, Ara Nefian, 
``Outlier removal in stereo reconstructions of orbital images", 
International Symposium on Visual Computing, 2010.

\end{outerlist}


%**************************************************************
%                      EDUCATION
%**************************************************************
\section{Education}


\href{http://www.unr.edu/}{\textbf{University of Nevada, Reno}},
Reno, NV 
\begin{outerlist}
\item[] Bachelors of Science, 
        \href{http://www.cse.unr.edu/}
             {Computer Science}
             (Graduated: May 2012)
        \begin{innerlist}
        \item UNR GPA: \emph{3.65}, CS GPA: \emph{3.6}
        \item Adviser:
              \href{http://www.cse.unr.edu/~bebis/}
                   {Dr. George Bebis}
        \item Areas of Interest: Computer Vision, Image Processing
        \item Specialized Undergraduate Courses: Computer Vision, Advanced Computer Vision, 
              Artificial Intelligence, Simulation Physics, and Image Processing.
        \item Graduate Courses Completed: Machine Learning, Computer Graphics, Patent Law (Business School).\\
        \end{innerlist}
\end{outerlist}

%UNITED STATES AIR FORCE
\href{http://www.af.mil/}{\textbf{Nevada Air National Guard}}
152 Maintenance Squadron, 152 Air Wing, Reno, NV
\begin{outerlist}
\item[] 2A553B, {Instrumentation and Flight Controls Journeyman}
        \begin{innerlist}
        \item Instrumentation and Flight Controls Craftsman Course
        \item Airman Leadership School, Correspondance, May 2011
        \item Instrumentation and Flight Controls Apprentice School,\\
        \href{http://www.sheppard.af.mil/}
             {Sheppard Air Force Base, TX}, June 2007
        \item Electronic Principles School,\\
        \href{http://www.keesler.af.mil/}
             {Keesler Air Force Base, MS}, January 2007
        \item Basic Military Training,\\
        \href{http://www.lackland.af.mil/}
             {Lackland Air Force Base, TX}, November 2006\\
        \end{innerlist}
\end{outerlist}


%**************************************************************
%                   Volunteering
%**************************************************************
\section{Volunteering}
\textbf{Coach, Codebusters First Tech Challenge Robotics Team}\\
    Boys and Girls Club of Truckee Meadows\\
    May 2016 - February 2020\\

\textbf{Mentor, Boys and Girls Club of Truckee Meadows}\\
    March 2015 - May 2016

%**************************************************************
%                   REFERENCES 
%**************************************************************
\section{References}
Available Upon Request
%\begin{itemize}
%\item Samuel Hancock\\
%      Technical Director, Sierra Nevada Corporation\\
%      Phone: 775-849-6244
%\item Dr. George Bebis\\
%      Chair, Department of Computer Science and UNR Computer Vision Lab Director\\
%      Phone: 775-784-6463
%\item Dr. Ara Nefian\\
%      Senior Research Scientist, NASA Ames Research Center\\
%      Phone: 650-604-0845
%\item Technical Sergeant Rebecca Varnum\\
%      Guidance and Control Shop Supervisor, 152 MXS, NVANG\\
%      Phone: 775-788-4565
%\end{itemize}
\end{document}

%%%%%%%%%%%%%%%%%%%%%%%%%% End CV Document %%%%%%%%%%%%%%%%%%%%%%%%%%%%%
